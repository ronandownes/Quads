\documentclass[a4paper,12pt]{article}
\usepackage{amsmath}
\usepackage{exsheets} % Import the exsheets package for creating exercises
\usepackage{geometry}
\geometry{margin=1in}
\usepackage{lmodern} % Enhanced font readability

% Setup for exsheets
\SetupExSheets{
  headings=runin,
  solution/print=true, % Uncomment this line if you want solutions printed
  solution/pre-hook={\par\textbf{Solution:}~}
}

\title{LC Ordinary Level Algebra}
\author{Ronan}
\date{\today}

\begin{document}

\maketitle

\section*{Introduction}
This worksheet covers essential topics in Ordinary Level Leaving Certificate Algebra. Each question is designed to help students build confidence and proficiency in algebraic manipulation and problem-solving.

\section{Linear Equations}
\begin{question}
Solve the following linear equations for \( x \):
\begin{enumerate}
    \item \( 2x + 5 = 15 \)
    \item \( 4x - 3 = 13 \)
    \item \( \frac{x}{2} + 7 = 10 \)
\end{enumerate}
\end{question}

\begin{solution}
\begin{enumerate}
    \item \( 2x + 5 = 15 \Rightarrow 2x = 10 \Rightarrow x = 5 \)
    \item \( 4x - 3 = 13 \Rightarrow 4x = 16 \Rightarrow x = 4 \)
    \item \( \frac{x}{2} + 7 = 10 \Rightarrow \frac{x}{2} = 3 \Rightarrow x = 6 \)
\end{enumerate}
\end{solution}

\section{Quadratic Equations}
\begin{question}
Solve the following quadratic equations:
\begin{enumerate}
    \item \( x^2 - 5x + 6 = 0 \)
    \item \( x^2 + 4x - 12 = 0 \)
\end{enumerate}
\end{question}

\begin{solution}
\begin{enumerate}
    \item \( x^2 - 5x + 6 = 0 \Rightarrow (x - 2)(x - 3) = 0 \Rightarrow x = 2, x = 3 \)
    \item \( x^2 + 4x - 12 = 0 \Rightarrow (x + 6)(x - 2) = 0 \Rightarrow x = -6, x = 2 \)
\end{enumerate}
\end{solution}

\section{Simultaneous Equations}
\begin{question}
Solve the following simultaneous equations:
\begin{enumerate}
    \item \( 2x + y = 7 \)
    \item \( 3x - y = 8 \)
\end{enumerate}
\end{question}

\begin{solution}
Adding the two equations: \\
\( (2x + y) + (3x - y) = 7 + 8 \Rightarrow 5x = 15 \Rightarrow x = 3 \).\\
Substitute \( x = 3 \) into \( 2x + y = 7 \Rightarrow 2(3) + y = 7 \Rightarrow y = 1 \).\\
Thus, the solution is \( x = 3, y = 1 \).
\end{solution}

\section{Algebraic Fractions}
\begin{question}
Simplify the following algebraic fractions:
\begin{enumerate}
    \item \( \frac{2x^2 - 8}{4x} \)
    \item \( \frac{x^2 - 9}{x + 3} \)
\end{enumerate}
\end{question}

\begin{solution}
\begin{enumerate}
    \item \( \frac{2x^2 - 8}{4x} = \frac{2(x^2 - 4)}{4x} = \frac{2(x - 2)(x + 2)}{4x} = \frac{(x - 2)(x + 2)}{2x} \)
    \item \( \frac{x^2 - 9}{x + 3} = \frac{(x - 3)(x + 3)}{x + 3} = x - 3 \) (for \( x \neq -3 \))
\end{enumerate}
\end{solution}

\section*{Conclusion}
This worksheet covered a range of algebraic topics. Students should ensure they understand each solution step-by-step before moving on to more complex problems.

\end{document}
