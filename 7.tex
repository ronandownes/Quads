\documentclass[12pt, a4paper, addpoints]{exam}

\usepackage{amsmath} % For mathematical symbols
\usepackage{multicol} % For multi-column layout
\usepackage{pgfmath} % For math parsing and calculations

% Define the \expandbinomials command
\newcommand{\expandbinomials}[4]{%
    % Compute intermediate terms (A, B, C)
    \pgfmathparse{int(#1*#3)} \let\A\pgfmathresult  % A = a*c
    \pgfmathparse{int(#1*#4 + #2*#3)} \let\B\pgfmathresult  % B = a*d + b*c
    \pgfmathparse{int(#2*#4)} \let\C\pgfmathresult  % C = b*d

    % Compute absolute values and cast to integer
    \pgfmathparse{int(abs(\A))} \let\absA\pgfmathresult
    \pgfmathparse{int(abs(\B))} \let\absB\pgfmathresult
    \pgfmathparse{int(abs(\C))} \let\absC\pgfmathresult

    % Handle the signs for A, B, C
    \ifnum\A>0 \def\signA{} \else \def\signA{-} \fi
    \ifnum\B>0 \def\signB{+} \else \def\signB{-} \fi
    \ifnum\C>0 \def\signC{+} \else \def\signC{-} \fi

    % Print the quadratic expression in math mode
    \[
    \signA
    \ifnum\A=1
        x^2
    \else
        \pgfmathprintnumber{\A}x^2
    \fi
    \ifnum\B=0
    \else
        \signB
        \ifnum\absB=1
            x
        \else
            \pgfmathprintnumber{\absB}x
        \fi
    \fi
    \ifnum\C=0
    \else
        \signC\pgfmathprintnumber{\absC}
    \fi
    \]
    
}

\begin{document}

\begin{questions}
\LARGE
\question Factor the following quadratic expressions.

\setlength{\columnsep}{50pt}
\begin{multicols}{2}
\begin{parts}

\part \expandbinomials{1}{3}{11}{5}
\part \expandbinomials{7}{11}{1}{2}
\part \expandbinomials{3}{5}{1}{11}
\part \expandbinomials{11}{1}{3}{7}
\part \expandbinomials{5}{1}{7}{3}
\part \expandbinomials{1}{11}{3}{5}
\part \expandbinomials{3}{7}{1}{11}
\part \expandbinomials{11}{3}{1}{5}
\part \expandbinomials{7}{1}{5}{3}
\part \expandbinomials{1}{5}{11}{3}

\end{parts}
\end{multicols}

\end{questions}

\end{document}
