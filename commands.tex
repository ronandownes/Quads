
\newcommand{\smallvspace}{\vspace{5mm}}
\newcommand{\normalvspace}{\vspace{10mm}}
\newcommand{\largevspace}{\vspace{15mm}}
\newcommand{\Largevspace}{\vspace{25mm}}
\newcommand{\LARGEvspace}{\vspace{55mm}}
\newcommand{\hugevspace}{\vspace{45mm}}
\newcommand{\Hugevspace}{\vspace{55mm}}
\newcommand{\twoup}{\vspace{75mm}}







\newcommand{\expandbinomials}[4]{%
    % Compute intermediate terms (A, B, C)
    \pgfmathparse{int(#1*#3)} \let\A\pgfmathresult  % A = a*c
    \pgfmathparse{int(#1*#4 + #2*#3)} \let\B\pgfmathresult  % B = a*d + b*c
    \pgfmathparse{int(#2*#4)} \let\C\pgfmathresult  % C = b*d

    % Compute absolute values and cast to integer
    \pgfmathparse{int(abs(\A))} \let\absA\pgfmathresult
    \pgfmathparse{int(abs(\B))} \let\absB\pgfmathresult
    \pgfmathparse{int(abs(\C))} \let\absC\pgfmathresult

    % Handle the signs for A, B, C
    \ifnum\A>0 \def\signA{} \else \def\signA{-} \fi
    \ifnum\B>0 \def\signB{+} \else \def\signB{-} \fi
    \ifnum\C>0 \def\signC{+} \else \def\signC{-} \fi

    % Print the quadratic expression in math mode
    \[
    \signA
    \ifnum\A=1
        x^2
    \else
        \pgfmathprintnumber{\A}x^2
    \fi
    \ifnum\B=0
    \else
        \signB
        \ifnum\absB=1
            x
        \else
            \pgfmathprintnumber{\absB}x
        \fi
    \fi
    \ifnum\C=0
    \else
        \signC\pgfmathprintnumber{\absC}
    \fi
    \]  

}


\newcommand{\productbinomials}[4]{%
    \[
    \left(
    \ifnum#1=1
        x
    \else
        \pgfmathprintnumber{#1}x
    \fi
    \ifnum#2>0
        +\pgfmathprintnumber{#2}
    \else\ifnum#2<0
        -\pgfmathprintnumber{-#2}
    \fi\fi
    \right)
    \left(
    \ifnum#3=1
        x
    \else
        \pgfmathprintnumber{#3}x
    \fi
    \ifnum#4>0
        +\pgfmathprintnumber{#4}
    \else\ifnum#4<0
        -\pgfmathprintnumber{-#4}
    \fi\fi
    \right)
    = 
    \pgfmathparse{#1*#3} \pgfmathprintnumber{\pgfmathresult}x^2
    \pgfmathparse{#1*#4 + #2*#3}
    \ifnum\pgfmathresult>0
        +\pgfmathprintnumber{\pgfmathresult}x
    \else
        \pgfmathprintnumber{\pgfmathresult}x
    \fi
    \pgfmathparse{#2*#4}
    \ifnum\pgfmathresult>0
        +\pgfmathprintnumber{\pgfmathresult}
    \else
        \pgfmathprintnumber{\pgfmathresult}
    \fi
    \]
}



\newcommand{\expandbinomialsequation}[4]{%
    % Compute intermediate terms (A, B, C)
    \pgfmathparse{int(#1*#3)} \let\A\pgfmathresult  % A = a*c
    \pgfmathparse{int(#1*#4 + #2*#3)} \let\B\pgfmathresult  % B = a*d + b*c
    \pgfmathparse{int(#2*#4)} \let\C\pgfmathresult  % C = b*d

    % Compute absolute values and cast to integer
    \pgfmathparse{int(abs(\A))} \let\absA\pgfmathresult
    \pgfmathparse{int(abs(\B))} \let\absB\pgfmathresult
    \pgfmathparse{int(abs(\C))} \let\absC\pgfmathresult

    % Handle the signs for A, B, C
    \ifnum\A>0 \def\signA{} \else \def\signA{-} \fi
    \ifnum\B>0 \def\signB{+} \else \def\signB{-} \fi
    \ifnum\C>0 \def\signC{+} \else \def\signC{-} \fi

    % Print the quadratic expression in math mode with "= 0"
    \[
    \signA
    \ifnum\A=1
        x^2
    \else
        \pgfmathprintnumber{\A}x^2
    \fi
    \ifnum\B=0
    \else
        \signB
        \ifnum\absB=1
            x
        \else
            \pgfmathprintnumber{\absB}x
        \fi
    \fi
    \ifnum\C=0
    \else
        \signC\pgfmathprintnumber{\absC}
    \fi
    = 0
    \]   
}
