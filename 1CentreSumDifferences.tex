\documentclass[12pt, a4paper, addpoints]{exam}
\pagestyle{empty} % Suppress page numbers

\usepackage[margin=15mm]{geometry} % Set all margins to 15mm
\usepackage{amsmath} % For mathematical symbols
\usepackage{multicol} % For multi-column layout
\usepackage{pgfmath} % For math parsing and calculations
\usepackage{xcolor} % For color customization
\usepackage[margin=15mm]{geometry}
% Define the \expandbinomials command
\input{commands}





% Suppress part numbering
% \renewcommand{\thepartno}{}




% Make part labels white (invisible on white background)


\begin{document}
\section{Trivial Sum and Difference type factors of  $Ax^2 + Bx +C$}. In each case below Solve for the roots by factors  and sketch the graphs. 

\begin{questions}
\LARGE

\question When adding outers gives middle.

\setlength{\columnsep}{20pt}
\begin{multicols}{2}
\begin{parts}
% Answer all \numparts   parts.
\part \expandbinomials{1}{1}{2}{3}  \vspace{45mm}
\part \expandbinomials{1}{1}{2}{5}  \vspace{45mm}
\part \expandbinomials{1}{1}{2}{7}  \vspace{45mm}
\part \expandbinomials{1}{1}{2}{11} \vspace{45mm}
\part \expandbinomials{1}{1}{2}{13} \vspace{45mm}
\part \expandbinomials{1}{1}{3}{5} \vspace{45mm}

\end{multicols}
\end{parts}

\question Notice in the last question  the outer coefficients  sum to the central coefficient. Put where is the $x^2$ coeifficient in your roots? How about the constant coefficient? Hint: Use the words numerator and denominator in your answers.




\newpage


\question Sum outers and negate for middle.   
\setlength{\columnsep}{20pt}
\begin{multicols}{2}
\begin{parts}
\part \expandbinomials{1}{-1}{3}{-7}  \vspace{45mm}
\part \expandbinomials{1}{-1}{3}{-11} \vspace{45mm}
\part \expandbinomials{1}{-1}{3}{-13} \vspace{45mm}
\part \expandbinomials{1}{-1}{5}{-7} \vspace{45mm}
\part \expandbinomials{1}{-1}{5}{-11} \vspace{45mm}
\part \expandbinomials{1}{-1}{5}{-13} \vspace{45mm}
% \part \expandbinomials{1}{1}{7}{11} 
% \part \expandbinomials{1}{1}{7}{13} 
% \part \expandbinomials{1}{1}{11}{13}
\end{parts}
\end{multicols}
\question In each case above both roots  have the same sign? Are they both positive or both negative? What sign pattern quadratic gives two positive roots?
\newpage

\question Central Difference Cases. Solve for the roots by factors  and sketch the graphs. 
\setlength{\columnsep}{20pt}
\begin{multicols}{3}
\begin{parts}
\part \expandbinomials{1}{1}{3}{-7}  \vspace{45mm}
\part \expandbinomials{1}{1}{3}{-11} \vspace{45mm}
\part \expandbinomials{1}{1}{3}{-13} \vspace{45mm}
\part \expandbinomials{1}{1}{5}{-7} \vspace{45mm}
\part \expandbinomials{1}{1}{5}{-11} \vspace{45mm}
\part \expandbinomials{1}{1}{5}{-13} \vspace{45mm}
\part \expandbinomials{1}{-1}{7}{11} \vspace{45mm}
\part \expandbinomials{1}{-1}{7}{13} \vspace{45mm}
\part \expandbinomials{1}{-1}{11}{13}\vspace{45mm}
\end{parts}
\end{multicols}
\question In each case above constant coefficient of the quadratic is negative? Are the  signs of the roots mixed or same?
\newpage

\question Central Difference Cases. Solve for the roots by factors  and sketch the graphs. 
\setlength{\columnsep}{20pt}
\begin{multicols}{3}
\begin{parts}
\part \expandbinomials{1}{1}{1}{-7}  \vspace{45mm}
\part \expandbinomials{1}{1}{1}{-6} \vspace{45mm}
\part \expandbinomials{1}{1}{1}{5} \vspace{45mm}
\part \expandbinomials{1}{1}{1}{-7} \vspace{45mm}
\part \expandbinomials{1}{1}{1}{3} \vspace{45mm}
\part \expandbinomials{1}{1}{1}{13} \vspace{45mm}
\part \expandbinomials{1}{-1}{1}{11} \vspace{45mm}
\part \expandbinomials{1}{-1}{1}{13} \vspace{45mm}
\part \expandbinomials{1}{-1}{1}{13}\vspace{45mm}
\end{parts}
\end{multicols}

\newpage



\question In these $x^2$ with positive constant cases numerical factoring is not needed.
\setlength{\columnsep}{20pt}
\begin{multicols}{3}
\begin{parts}
\part \expandbinomials{1}{1}{1}{17}  \vspace{45mm}
\part \expandbinomials{1}{1}{1}{19} \vspace{45mm}
\part \expandbinomials{1}{1}{1}{20} \vspace{45mm}
\part \expandbinomials{1}{1}{1}{29} \vspace{45mm}

\end{parts}
\end{multicols}


\question Simplest  ($x^2$) with positive constant case.
\setlength{\columnsep}{20pt}
\begin{multicols}{3}
\begin{parts}
\part \expandbinomials{5}{1}{1}{-1}  \vspace{45mm}
\part \expandbinomials{11}{-1}{1}{1} \vspace{45mm}
\part \expandbinomials{42}{1}{1}{-1}  \vspace{45mm}
\part \expandbinomials{8}{1}{1}{-1} \vspace{45mm}
\end{parts}
\end{multicols}
\newpage

















\question Simplest  $x^2$ with negative constant case in where improper factors are needed? 
\setlength{\columnsep}{20pt}
\begin{multicols}{3}
\begin{parts}
\part \expandbinomials{1}{1}{1}{-7}  \vspace{45mm}
\part \expandbinomials{1}{1}{1}{-6} \vspace{45mm}
\part \expandbinomials{1}{1}{1}{-30} \vspace{45mm}
\part \expandbinomials{1}{1}{1}{-37} \vspace{45mm}

\end{parts}
\end{multicols}


\question  Simplest case for $-1$ constant. 
\setlength{\columnsep}{20pt}
\begin{multicols}{3}
\begin{parts}
\part \expandbinomials{5}{1}{1}{-1}  \vspace{45mm}
\part \expandbinomials{11}{-1}{1}{1} \vspace{45mm}
\part \expandbinomials{42}{1}{1}{-1}  \vspace{45mm}
\part \expandbinomials{8}{1}{1}{-1} \vspace{45mm}

\end{parts}
\end{multicols}
\newpage



\end{questions}

\end{document}
