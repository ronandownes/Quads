\documentclass[12pt, a4paper, addpoints]{exam}
\pagestyle{empty} % Suppress page numbers

\usepackage[margin=15mm]{geometry} % Set all margins to 15mm
\usepackage{amsmath} % For mathematical symbols
\usepackage{multicol} % For multi-column layout
\usepackage{pgfmath} % For math parsing and calculations
\usepackage{xcolor} % For color customization
\usepackage[margin=15mm]{geometry}
% Define the \expandbinomials command
\input{commands}



% Redefine the \part command to avoid a line break after the part number
\renewcommand{\partlabel}{\thepartno)~}


% Suppress part numbering
% \renewcommand{\thepartno}{}




% Make part labels white (invisible on white background)


\begin{document}
\section{Sum and difference central coeifficients}

\begin{questions}
\LARGE

\question Solve for the roots by factors  and sketch the graphs. 

\setlength{\columnsep}{20pt}
\begin{multicols}{2}
\begin{parts}
% Answer all \numparts   parts.
\part \productbinomials{1}{1}{2}{3}  \vspace{45mm}
\part \expandbinomials{1}{1}{2}{5}  \vspace{45mm}
\part \expandbinomials{1}{1}{2}{7}  \vspace{45mm}
\part \expandbinomials{1}{1}{2}{11} \vspace{45mm}
\part \expandbinomials{1}{1}{2}{13} \vspace{45mm}
\part \expandbinomials{1}{1}{3}{5} \vspace{45mm}

\end{multicols}
\end{parts}

\question Notice in the last question  the outer coefficients  sum to the central coefficient. Put where is the $x^2$ coeifficient in your roots? How about the constant coefficient? Hint: Use the words numerator and denominator in your answers.




\newpage


\question Central Sum negated cases. Solve for the roots by factors  and sketch the graphs. 
\setlength{\columnsep}{20pt}
\begin{multicols}{3}
\begin{parts}
\part \expandbinomials{1}{-1}{3}{-7}  \vspace{45mm}
\part \expandbinomials{1}{-1}{3}{-11} \vspace{45mm}
\part \expandbinomials{1}{-1}{3}{-13} \vspace{45mm}
\part \expandbinomials{1}{-1}{5}{-7} \vspace{45mm}
\part \expandbinomials{1}{-1}{5}{-11} \vspace{45mm}
\part \expandbinomials{1}{-1}{5}{-13} \vspace{45mm}
% \part \expandbinomials{1}{1}{7}{11} 
% \part \expandbinomials{1}{1}{7}{13} 
% \part \expandbinomials{1}{1}{11}{13}
\end{parts}
\end{multicols}
\question In each case above both roots  have the same sign? Are they both positive or both negative? What kind of quadratic gives two positive roots?
\newpage

\question Central Difference Cases. Solve for the roots by factors  and sketch the graphs. 
\setlength{\columnsep}{20pt}
\begin{multicols}{3}
\begin{parts}
\part \expandbinomials{1}{1}{3}{-7}  \vspace{45mm}
\part \expandbinomials{1}{1}{3}{-11} \vspace{45mm}
\part \expandbinomials{1}{1}{3}{-13} \vspace{45mm}
\part \expandbinomials{1}{1}{5}{-7} \vspace{45mm}
\part \expandbinomials{1}{1}{5}{-11} \vspace{45mm}
\part \expandbinomials{1}{1}{5}{-13} \vspace{45mm}
\part \expandbinomials{1}{-1}{7}{11} \vspace{45mm}
\part \expandbinomials{1}{-1}{7}{13} \vspace{45mm}
\part \expandbinomials{1}{-1}{11}{13}\vspace{45mm}
\end{parts}
\end{multicols}
\question In each case above constant coefficient of the quadratic is negative? What can we say about the signs of the roots?
\newpage



\end{questions}

\end{document}
